\documentclass{article}

\usepackage[utf8]{inputenc}
\usepackage[T1]{fontenc}
\usepackage[letterpaper, margin=1in, bottom=1in]{geometry}
\usepackage[authoryear,sort,square]{natbib}
\usepackage{amsmath}
\usepackage{amssymb}
\usepackage{amsthm}
\usepackage{bbm}
\usepackage{bbold}
\usepackage{stmaryrd}
\usepackage{mathtools}
\usepackage{mathpartir}
\usepackage[dvipsnames]{xcolor}
\usepackage{latex-pl-syntax/pl-syntax}
\usepackage{xspace}
\usepackage{suffix}
\usepackage{turnstile}
\usepackage{multicol}
\usepackage{fontawesome}

\usepackage[hidelinks]{hyperref}

% Notes
\newcommand{\uncertain}[1]{{\color{red} #1}\xspace}
\newcommand{\newcommenter}[3]{%
  \newcommand{#1}[1]{%
    \textcolor{#2}{\small\textsf{[{#3}: {##1}]}}%
  }%
}
\definecolor{darkgreen}{rgb}{0,0.7,0}
\newcommenter{\akh}{purple}{AKH}

% AMSTHM Setup
\newtheorem{thm}{Theorem}
\newtheorem{lem}{Lemma}
\newtheorem{cor}{Corollary}
\newtheorem{conj}{Conjecture}
\newtheorem{inv}{Invariant}
\theoremstyle{definition}
\newtheorem{defn}{Definition}

% Macros

\newcommand{\uu}{\ensuremath{(\,)}}

\newcommand{\letl}[4]{\textsf{let}\;#1.#2 \mathrel{\coloneqq} #3 \mathrel{\textsf{in}} #4}

\newcommand{\fun}[3]{\textsf{fun}\;#1(#2) \mathrel{\coloneqq} #3}
\newcommand{\lam}[2]{\lambda#1.\;#2}
\newcommand{\appfun}[2]{#1\;#2}

\newcommand{\projname}{\textsf{proj}}
\newcommand{\projl}[1]{\appfun{\projname_1}{#1}}
\newcommand{\projr}[1]{\appfun{\projname_2}{#1}}

\newcommand{\inl}[1]{\appfun{\textsf{inl}}{#1}}
\newcommand{\inr}[1]{\appfun{\textsf{inr}}{#1}}
\newcommand{\case}[5]{\ensuremath{\textsf{case}\;#1 \mathrel{\textsf{of}} \inl{#2} \Rightarrow #3;\; \inr{#4} \Rightarrow #5}}

\newcommand{\casev}[1]{\ensuremath{\textsf{case}\;#1 \mathrel{\textsf{of}} \_ \Rightarrow \bot}}

\newcommand{\send}[2]{#1 \mathbin{\leadsto} #2}

\newcommand{\upname}{\textsf{up}}
\newcommand{\up}[2]{\appfun{\upname_{#1}}{#2}}
\newcommand{\downname}{\textsf{down}}
\newcommand{\down}[2]{\appfun{\downname_{#1}}{#2}}

\newcommand{\unit}{\textsf{unit}\xspace}
\newcommand{\void}{\textsf{void}\xspace}

\newcommand{\lock}{\text{\faLock}}
\newcommand{\at}{\ensuremath{\mathrel{@}}}
\newcommand{\locks}[1]{\text{locks}(#1)}

\newcommand{\reps}[2]{#1 \mathrel{\vartriangleright} #2}
\newcommand{\cansend}[2]{\text{CanSend}(#1, #2)}
\newcommand{\candown}[2]{\text{CanDown}(#1, #2)}
\newcommand{\canup}[2]{\text{CanUp}(#1, #2)}

\bibliographystyle{plainnat}

\begin{document}

\section{Syntax}
\label{sec:syntax}

\begin{syntax}
  \category[Types]{\tau}
    \alternative{\unit}
    \alternative{\void}
    \alternative{\ell.\tau}
    \alternative{\tau_1 \times \tau_2}
    \alternative{\tau_1 + \tau_2}
    \alternative{\tau_1 \to \tau_2}
  \category[Expressions]{e}
    \alternative{x}
    \alternative{\uu}
    \alternative{\ell.e}
    \alternative{\letl{\ell}{x}{e_1}{e_2}}\\
    \alternative{(e_1, e_2)}
    \alternative{\projl{e}}
    \alternative{\projr{e}}\\
    \alternative{\inl{e}}
    \alternative{\inr{e}}
    \alternative{\case{e_1}{x}{e_2}{y}{e_3}}\\
    \alternative{\casev{e}}\\
    \alternative{\lam{x}{e}}
    \alternative{\appfun{e_1}{e_2}}\\
    \alternative{\send{e}{\ell}}
    \alternative{\up{m}{e}}
    \alternative{\down{m}{e}}
  \category[Modalities]{m}
    \alternative{\diamond}
    \alternative{m \cdot \ell}
  \category[Contexts]{\Gamma}
    \alternative{\cdot}
    \alternative{\Gamma, x : \tau \at m}
    \alternative{\Gamma, \lock_m}
\end{syntax}

\section{Type System}
\label{sec:type-system}

\begin{mathpar}
  \infer{ }{x : \tau \at m \overset{\diamond}{\in} \Gamma, x : \tau \at m} \and
  \infer{x : \tau \at m \overset{m'}{\in} \Gamma}{x : \tau \at m \overset{m'}{\in} \Gamma, y : \sigma \at m''} \and
  \infer{x : \tau \at m \overset{m'}{\in} \Gamma}{x : \tau \at m \overset{m' \cdot m''}{\in} \Gamma, \lock_{m''}} \and
\end{mathpar}

$$
\locks{\Gamma} = \left\{\begin{array}{ll}
  \diamond & \text{if $\Gamma = \cdot$}\\
  \locks{\Delta} & \text{if $\Gamma = \Delta, x : \tau \at m$}\\
  \locks{\Delta} \cdot m & \text{if $\Gamma = \Delta, \lock_m$}
\end{array}
\right.
$$

\begin{lem}
  $x : \tau \at m \overset{m'}{\in}\Gamma$ if and only if $\Gamma = \Delta_1, x : \tau \at m, \Delta_2$ where $\locks{\Delta_2} = m'$.
\end{lem}

The type system is parameterized on several relations:
\begin{itemize}
\item $\reps{m}{m'}$
\item $\cansend{m}{m'}$
\item $\canup{m}{m'}$
\item $\candown{m}{m'}$
\end{itemize}
These control when certain rules are available.
Different settings result in different interpretations of the calculus.

\begin{mathpar}
  \infer{
    x : \tau \at m \overset{m'}{\in} \Gamma\\
    \reps{m}{m'}
  } {
    \Gamma \vdash x : \tau
  }\and
  \infer {  } {
    \Gamma \vdash \uu : \unit
  }\and
  \infer{
  \Gamma, \lock_\ell \vdash e : \tau
}{
  \Gamma \vdash \ell.e : \ell.\tau
}\and
\infer{
  \Gamma \vdash e_1 : \ell.\tau\\
  \Gamma, x : \tau \at \ell \vdash e_2 : \sigma
}{
  \Gamma \vdash \letl{\ell}{x}{e_1}{e_2}
}\\
\infer{
  \Gamma \vdash e_1 : \tau_1\\
  \Gamma \vdash e_2 : \tau_2
}{
  \Gamma \vdash (e_1, e_2) : \tau_1 \times \tau_2
} \and
\infer{
  \Gamma \vdash e : \tau_1 \times \tau_2
}{
  \Gamma \vdash \projl{e} : \tau_1
} \and
\infer{
  \Gamma \vdash e : \tau_1 \times \tau_2
}{
  \Gamma \vdash \projr{e} : \tau_2
} \\
\infer{
  \Gamma \vdash e : \tau_1
}{
  \Gamma \vdash \inl{e} : \tau_1 + \tau_2
}\and
\infer{
  \Gamma \vdash e : \tau_2
}{
  \Gamma \vdash \inr{e} : \tau_1 + \tau_2
}\and
\infer{
  \Gamma \vdash e_1 : \tau_1 + \tau_2\\
  \Gamma, x : \tau_1 \at \diamond \vdash e_2 : \sigma\\
  \Gamma, y : \tau_2 \at \diamond \vdash e_3 : \sigma
}{
  \Gamma \vdash \case{e_1}{x}{e_2}{y}{e_3} : \sigma
}\\
\infer{
  \Gamma \vdash e : \void
}{
  \Gamma \vdash \casev{e} : \tau
}\and
\infer{
  \Gamma, x : \tau \at \diamond \vdash e : \sigma
}{
  \Gamma \vdash \lam{x}{e} : \tau \to \sigma
}\and
\infer{
  \Gamma \vdash e_1 : \tau \to \sigma\\
  \Gamma \vdash e_2 : \tau
}{
  \Gamma \vdash \appfun{e_1}{e_2} : \sigma
}\\
\infer{
  \Gamma \vdash e : m.\tau\\
  \cansend{m}{m'}
}{
  \Gamma \vdash \send{e}{m'} : m'.\tau
}\and
\infer{
  \Gamma \vdash e : \tau\\
  \canup{\locks{\Gamma}}{m}
}{
  \Gamma \vdash \up{m}{e} : m.\tau
}\and
\infer{
  \Gamma \vdash e : m.\tau\\
  \candown{\locks{\Gamma}}{m}
}{
  \Gamma \vdash \down{m}{e} : \tau
}
  
  
\end{mathpar}

\section{Multitier Interpretation}
\label{sec:mult-interpr}

\begin{mathpar}
  \left\llbracket \infer{\Gamma, \lock_\ell \vdash e : \tau}{\Gamma \vdash \ell.e : \ell.\tau} \right\rrbracket_{\locks{\Gamma}\cdot\ell} = \textsf{send}\;\llbracket \Gamma, \lock_\ell \vdash e : \tau \rrbracket_{\locks{\Gamma} \cdot \ell} \mathrel{\textsf{to}} \locks{\Gamma}\\
  \left\llbracket \infer{\Gamma, \lock_\ell \vdash e : \tau}{\Gamma \vdash \ell.e : \ell.\tau} \right\rrbracket_{\locks{\Gamma}} = \textsf{wrap}\;\ell\; (\textsf{receive}\;\textsf{from}\;\locks{\Gamma}\cdot \ell)\\
  \left\llbracket\infer{\Gamma \vdash e_1 : \ell.\tau\\ \Gamma, x : \tau \at \ell \vdash e_2 : \sigma}{\Gamma \vdash \letl{\ell}{x}{e_1}{e_2} : \sigma}\right\rrbracket_{\locks{\Gamma}\cdot\ell} =
  \textsf{let}\; x = \textsf{receive} \mathrel{\textsf{from}} \locks{\Gamma} \mathrel{\textsf{in}} \llbracket \Gamma, x : \tau @ \ell \vdash e_2 : \sigma \rrbracket_{\locks{\Gamma} \cdot \ell}\\
  \left\llbracket\infer{\Gamma \vdash e_1 : \ell.\tau\\ \Gamma, x : \tau \at \ell \vdash e_2 : \sigma}{\Gamma \vdash \letl{\ell}{x}{e_1}{e_2} : \sigma}\right\rrbracket_{\locks{\Gamma}} =
  \textsf{send}(\textsf{unwrap}\;\ell\;\llbracket\Gamma \vdash e_1 : \ell.\tau\rrbracket_{\locks{\Gamma}}); \llbracket \Gamma, x : \tau @ \ell \vdash e_2 : \sigma \rrbracket
\end{mathpar}

\section{Choreographic Interpretation}
\label{sec:chor-interpr}

\section{Properties}
\label{sec:properties}

We want to show the properties; where appropriate, we want to show them for every interpretation we discuss.

\begin{thm}[Normalization]
  If $\Gamma \vdash e : \tau$ then there is a trace~$t$ and term~$n$ such that $e \xRightarrow{t} n$ and $\vdash_{nf} n$ and $\Gamma \vdash n : \tau$.
\end{thm}

\begin{thm}[Normal Form Noninterference]
  If\/ $\Gamma, x : \tau \at m \vdash e : m'.\sigma$, $\lnot (\reps{m}{m'})$, $\lnot\canup{m}{m'}$, $\lnot\candown{m}{m'}$, and $\lnot\cansend{m}{m'}$, then if\/ $\vdash_{nf} e$ then $x \not\in \text{FV}(e)$.
\end{thm}

\begin{cor}[Noninterference]
  If\/ $\Gamma, x : \tau \at m \vdash e_1 : m'.\sigma$ and $\Gamma \vdash e_2, e_3 : m.\tau$ and there is no trust relationship between~$m$ and~$m'$ (as described above), then $\letl{m}{x}{e_2}{e_1} \equiv_{\alpha, \beta, \eta} \letl{m}{x}{e_3}{e_1}$.
\end{cor}

\begin{thm}[Correctness of EPP]
  If\/ $\Gamma \vdash e : \tau$ and $e \xRightarrow{t} n$ where\/ $\vdash_{nf} n$, then $\llbracket e \rrbracket_m \xRightarrow{\llbracket t \rrbracket_m} \llbracket n \rrbracket_m$.
\end{thm}

\begin{cor}[Deadlock Freedom]
  If\/ $\Gamma \vdash e : \tau$ then there is a normal form~$n$ and trace~$t$ such that $\llbracket e \rrbracket \xRightarrow{t} n$
\end{cor}

\end{document}
%%% Local Variables:
%%% mode: latex
%%% TeX-master: t
%%% End:
