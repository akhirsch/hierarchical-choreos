\section{Introduction}
\label{sec:introduction}

Choreographic and multitier programming model concurrent systems as containing a set number of interacting processes.
Intuitively, these processes act as interacting agents, such as nodes in a distributed system.
However, propositions-as-types models of choreographies~\cite{CarboneMS14} do not respect this structure, instead exploiting the linear structure of communication implicit in choreographies.
The recent proliferation of \emph{functional} choreographic programming~\cite{HirschG22,GraversenHM23,CruzFilipeGLMP21}, however, suggests a different potential propositions-as-types interpretation: a non-linear logic with explicit agents.

\emph{Doxastic logic}, the logic of beliefs, seems like a natural fit.
Doxastic logics represent agents as modalities of belief.
If the agent~$A$ believes the formula~$\varphi$, then $\square_A \varphi$ holds.
Doxastic logics represent \emph{belief} rather than \emph{knowledge}, so $\square_A \varphi$ holding does not necessarily imply that $\varphi$ holds as this would imply that an agent is correct about all of their beliefs.
This is the key difference between doxastic and \emph{epistemic} logics.

In order for doxastic logics to serve as the other side of this propositions-as-types correspondence, we need to interpret formulae as types and proofs as data.
We propose that most connectives have their standard intuitionistic interpretations, while we interpret $\square_A \varphi$ as saying that the process~$A$ has a $\varphi$.
This fits with doxastic logic, rather than epistemic logic, as described above.

However, there one major component of doxastic logics does not fit naturally into the global concurrent programming paradigm: nested beliefs.
For instance, the formula $\mathop{\square_A} (\mathop{\square_B} \varphi)$ represents ``agent~$A$ believes that agent~$B$ believes~$\varphi$.''
Current functional choreographic languages do not allow nested claims of ownership over data.%
\footnote{PolyChor$\lambda$~\cite{GraversenHM23} uses a similar notion to represent delegation, though this does not seem to follow the logical laws of doxastic logic.
Meanwhile, in unpublished work \citeauthor{Kavvos24} has been exploring an interpretation of doxastic logic where agent~$A$ has a remote pointer to $B$'s data~\cite{Kavvos24}.}
For instance, in Pirouette, one of the first functional choreographic programming languages~\cite{HirschG22}, local data have types which are disjoint from choreographic data.
Indeed, it is not entirely clear what it would mean for agent~$A$ to have beliefs about agent~$B$.
However, note that because $\square_A$ represents a belief of $A$'s, there is no requirement that $A$'s beliefs about $B$ have any connection to $B$ at all.
Thus, $A$'s version of $B$ can be viewed as its own, independent agent.

With this view, agents are arranged in a forest topology, with each agent ``above'' its parent.
While this breaks with global-view tradition, where processes have a fully connected topology where anyone can talk to anyone else, this hierarchical topology for processes is far from unnatural.
For instance, in a distributed system where nodes can themselves be multithreaded, each node~$A$ can be represented by an agent, with each thread on~$A$ represented as an agent above $A$ in the tree.

It would not make sense for nodes to be unable to communicate with one another, so we would like to be able to communicate between processes on the same level.
However, no node~$A$ should be able to talk to another node~$B$'s thread without going through~$B$ itself.
We model this as allowing communication between sibling nodes.
This complicates the simple forest topology described earlier, since the children of every node form a fully connected topology.
In other words, each node can communicate with its parent, its children, \emph{and} its siblings.
We dub choreographic programming with this ``tree of interconnected processes'' topology \emph{hierarchical choreographic programming}.

This leads to \emph{authorization logic}, a version of doxastic logic with explicit communication.
However, authorization logic allows for more-subtle topologies by restricting which siblings may communicate.
In particular, authorization logic contains a new judgment---$A \mathrel{\textsf{speaksfor}} B$---representing the ability of agents~$A$ and~$B$, who are at the same level in the tree, to communicate.
Then, if $\square_A \varphi$ and $A \mathrel{\textsf{speaksfor}} B$, then $A$ can send its proof of $\varphi$ to $B$, allowing us to derive $\square_B \varphi$.
We can thus think of \textsf{speaksfor} as a one-way channel---since $A$ can send to $B$, but not vice versa---and \textsf{speaksfor}-elimination as the choreographic ``send-and-receive'' primitive.

This paper describes ongoing research into formalizing this connection between doxastic logics, especially authorization logics, and global concurrent programming.
First, we describe Corps, a core calculus for hierarchical choreographic programming, allowing us to form the tree topology described above with explicit intra-level communication.
We then provide a number of conjectures about Corps and its relationships to doxastic and authorization logics.
By providing an explicit propositions-as-types interpretation of authorization logic in choreographies, we hope to be able to advance both technologies considerably.

%%% Local Variables:
%%% mode: latex
%%% TeX-master: "conference"
%%% End:
