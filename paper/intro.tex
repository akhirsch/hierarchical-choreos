\section{Introduction}
\label{sec:introduction}

Choreographic and multitier programming model concurrent systems as containing a set number of interacting processes.
Intuitively, these processes act as interacting agents, such as nodes in a distributed system.
However, propositions-as-typed models of choreographies~\cite{CarboneMS14} do not respect this structure, instead exploiting the linear structure of communication implicitly in choreographies.
The recent proliferation of \emph{functional} choreographic programming~\cite{HirschG22,GraversenHM23,CruzFilipeGLMP21}, however, suggests a different potential propositions-as-types interpretation: a non-linear logic with explicit agents.
\emph{Doxastic logic}, the logic of beliefs, seems like a natural fit.

Doxastic logics represent agents as modalities of belief.
If the agent~$A$ believes the formula~$\varphi$, then $\square_A \varphi$ is true.
Because doxastic logics represent \emph{belief} rather than \emph{knowledge}, just because $\square_A \varphi$ is true does not mean that $\varphi$ is true---this would imply that an agent is correct about all of their beliefs.
In a propositions-as-types paradigm, we view formulae as types and proofs as data.
Thus, we can imagine interpreting $\square_A \varphi$ as saying that the process~$A$ has a $\varphi$.
Just because a particular agent has data of a certain type does not mean that this data is available globally, just as we can see in doxastic logics.
Thus, doxastic logics seem like a good candidate for a propositions-as-types model of global concurrent paradigms, such as choreographic and multitier programming.

However, there is one major component of doxastic logics that do not fit naturally into the global concurrent programming paradigm: beliefs about beliefs.
For instance, the formula $\square_A (\square_B \varphi)$ represents ``agent~$A$ believes that agent~$B$ believes~$\varphi$.''
Current functional choreographic languages do not allow nested claims of ownership over data.
For instance, in Pirouette, one of the first functional choreographic programming languages~\cite{HirschG22}, local data have types which are disjoint from choreographic data.
Indeed, it is not entirely clear what it would mean for agent~$A$ to have beliefs about agent~$B$.
(Note that PolyChor$\lambda$~\cite{GraversenHM23} uses a similar notion to represent delegation, though this does not seem to follow the logical laws of doxastic logic.
Meanwhile, in unpublished work \citeauthor{Kavvos24} has been exploring an interpretation where agent~$A$ has a remote pointer to $B$'s data~\cite{Kavvos24}.)
However, note that because $\square_A$ represents a belief of $A$'s, there is no requirement that $A$'s beliefs about $B$ have any connection to $B$ at all.
Thus, $A$'s version of $B$ can be viewed as its own, independent agent.

% \begin{figure}
%   \centering
%   \begin{tikzpicture}
%     \node at (2, 0) (a) {$A$};
%     \node at (-2, 0) (b) {$B$};
%     \node at (3, 1) (c) {$C$};
%     \node at (1, 1) (d) {$D$};
%     \node at (-1, 1) (e) {$E$};
%     \node at (-3, 1) (f) {$F$};
%     \node at (-1, 2) (g) {$G$}
%     \draw (a) -> (c);
%     \draw (a) -> (d);
%     \draw (b) -> (e);
%     \draw (b) -> (f);
%     \draw (e) -> (g);
%   \end{tikzpicture}
  
%   \caption{Simple Forest Topology}
%   \label{fig:forest-topo}
% \end{figure}

With this view, agents are arranged in a forest topology, with each agent ``above'' its parent. %as visualized in Figure~\ref{fig:forest-topo}.
% Here, there are two nodes~$A$ and $B$ at the ``main'' level, each of which owns two nodes ($D$ and $C$ and $F$ and $E$, respectively).
% The node $E$ additionally owns the node $G$.
This breaks with global-view tradition, where processes have a connected topology---anyone can talk to anyone else.
Nevertheless, this hierarchical topology for processes is far from unnatural.
For instance, in a distributed system where nodes can themselves be multithreaded, each node~$A$ can be represented by an agent, with each thread~$B$ represented in an agent above $A$ in the tree.
This is essentially the core trick and insight of binary session types~\cite{Wadler12,DeYoungCPT12,CairesP10,DeYoungCPT09}.

However, we would like to be able to communicate between processes on the same level.
For instance, nodes~$A$ and~$B$ should be able to talk to each other, and each should be able to talk to any of their secondary threads.
However, node~$A$ should not be able to talk to node~$B$'s secondary thread without going through~$B$ itself.
This can be done through a choreographic send.
Now, the topology is not a simple forest topology: instead, there is a connected topology between the children of every node.
In other words, each node can communicate with its parent, its children, \emph{and} its siblings.

This leads to \emph{authorization logic}, a version of doxastic logic with explicit communication.
However, authorization logic allows for more-subtle topologies by restricting which siblings may communicate.
In particular, authorization logic contains a new judgment---$A \mathrel{\textsf{speaksfor}} B$---representing the ability of agents~$A$ and~$B$, who are at the same level in the tree, to communicate.
Then, if $\square_A \varphi$ and $A \mathrel{\textsf{speaksfor}} B$, then $A$ can send its proof of $\varphi$ to $B$, allowing us to derive $\square_B \varphi$.
We can thus think of \textsf{speaksfor} as a one-way channel---since $A$ can send to $B$, but not vice versa---and \textsf{speaksfor}-elimination as the choreographic ``send-and-receive'' primitive.

In this talk, I will discuss current, ongoing research into formalizing this connection between doxastic logics, especially authorization logics, and global concurrent programming.
First, I will describe a new calculus for hierarchical computing, allowing us to form the tree topology described above with explicit intra-level communication.
Then, I will describe a number of conjectures both on the logical side and on the computational side about these systems.
By providing an explicit propositions-as-types interpretation of authorization logic and choreographies, we hope to be able to advance both technologies considerably.

%%% Local Variables:
%%% mode: latex
%%% TeX-master: "conference"
%%% End:
