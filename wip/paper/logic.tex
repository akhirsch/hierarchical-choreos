\section{Background: Doxastic and Authorization Logics}
\label{sec:backgr-doxast-auth}

As described in Section~\ref{sec:introduction}, doxastic logics are modal logics of belief.
In this section, we present the main rules of a simple authorization logic in Fitch style~\cite{Clouston18}.
One rule requires reasoning about trust (modeled as a relation called \textsf{speaksfor}) and communication directly.
This rule separates authorization logics from other doxastic logics.

\begin{figure}
  \centering
  \begin{syntax}
    \abstractCategory[Agents]{A, B}
    \category[Generalized Agents]{g} \alternative{\diamond} \alternative{g \cdot A}
    \category[Formulae]{\varphi, \psi}
      \alternative{\top}
      \alternative{\bot}
      \alternative{\varphi \land \psi}
      \alternative{\varphi \lor \psi}
      \alternative{\varphi \to \psi}
      \alternative{\believes{A}{\varphi}}
    \category[Contexts]{\Gamma, \Delta}
      \alternative{\cdot}
      \alternative{\Gamma, \varphi \at g}
      \alternative{\Gamma, \lock_g}
  \end{syntax}
  
  \caption{Syntax of Doxastic Logic}
  \label{fig:dox-syn}
\end{figure}

The syntax of doxastic logic formulae can be found in Figure~\ref{fig:dox-syn}.
The syntax of the logic is parameterized on a set of \emph{agents} about which we wish to reason; we denote elements of this set with capital Latin characters like~$A$.
The formulae (denoted with lower-case Greek letters like~$\varphi$) mostly consist of those of intuitionistic propositional logic, with their usual meanings.
However, we also have the \emph{modality} $\believes{A}{}$, representing $A$'s beliefs.
Thus, we should read $\believes{A}{\varphi}$ as ``$A$ believes $\varphi$.''

The contexts (denoted by capital Greek letters like~$\Gamma$), however, are more unusual.
Most logics use lists of formulae for their contexts, and then use structural rules (or their equivalent) to force those lists to act as sets.
Here, we use sets of formulae tagged with \emph{generalized agents}, separated by similarly-tagged locks.
Intuitively, a generalized agent $g$ represents beliefs-of-beliefs: the generalized agent $\diamond$ is ground truth, $\diamond \cdot A$ is the agent $A$, $\diamond \cdot A \cdot B$ is $A$'s beliefs about $B$, and so on.
Thus, a generalized agent represents a path up the tree topology described in Section~\ref{sec:introduction}.
Then $\varphi \at g$ represents that $g$ believes $\varphi$---$\!\!{}\at \diamond \cdot A$ internalizes $\believes{A}{}$ in the same way that commas internalize conjunctions.
The context $\Gamma, \lock_g$ represents $g$'s view of the context $\Gamma$---$g$ can access its own beliefs, but not the beliefs of others.

Generalized agents are essentially (snoc) lists of agents, and thus form a monoid in a standard way.
We write the monoid action as $\join$.
In particular, $$g \join g' = \left\{\begin{array}{ll} g & \text{if $g' = \diamond$}\\ (g \join g'') \cdot A & \text{if $g' = g'' \cdot A$}\end{array}\right.$$
We can then compute the locks in a context.
We consider a judgment $\Gamma \vdash \varphi$ to be a proof of $\varphi$ ``from $\locks{\Gamma}$'s point of view.''
$$\locks{\Delta} = \left\{\begin{array}{ll} \diamond & \text{if $\Delta = \cdot$}\\ \locks{\Delta'} & \text{if $\Delta = \Delta', \varphi \at g$}\\ \locks{\Delta'} \join g & \text{if $\Delta = \Delta', \lock_g$}\end{array}\right.$$
Contexts also use the monoid action freely to reason about locks; in other words, contexts obey the equations $\Gamma, \lock_\diamond = \Gamma$ and $\Gamma, \lock_{g_1}, \lock_{g_2} = \Gamma, \lock_{g_1 \join g_2}$.

\begin{figure}
  \centering

  \begin{mathpar}
    \infer*[left=Axiom]{\locks{\Delta} = g }{\Gamma, \varphi \at g, \Delta \vdash \varphi} \\
    \infer*[left=BelievesI]{\Gamma, \lock_A \vdash \varphi}{\Gamma \vdash \believes{A}{\varphi}}\and
    \infer*[right=BelievesE]{\Gamma, \lock_{g_1} \vdash \believes{g_2}{\varphi} \\ \Gamma, \varphi \at g_1 \mathbin{+\!\!+} g_2 \vdash \psi}{\Gamma \vdash \psi} \and
    \infer*[left=SelfDown]{\Gamma \vdash \believes{A}{(\believes{A}{\varphi})}}{\Gamma \vdash \believes{A}{\varphi}} \and
    \infer*[right=SelfUp]{\Gamma \vdash \believes{A}{\varphi}}{\Gamma \vdash \believes{A}{(\believes{A}{\varphi})}}\\
    \infer*[right=SpeaksforE]{A \mathrel{\textsf{speaksfor}} B\\ \Gamma \vdash \believes{A}{\varphi}}{\Gamma \vdash \believes{B}{\varphi}}
  \end{mathpar}
  
  \caption{Selected Rules for Doxastic Logic and Authorization Logic}
  \label{fig:dox-rules}
\end{figure}

Figure~\ref{fig:dox-rules} contains selected rules of authorization logic.
(The rule \textsc{SpeaksforE} is a rule of authorization logic not found in standard doxastic logic.)
The rules presented here define the behavior of the $\believes{A}{}$ modality.
Only the standard rules of intuitionistic propositional logic remain.

The first rule is \textsc{Axiom}, which allows the use of an assumption~$\varphi \at g$.
In order to use such an assumption, we must know that the locks to the left of the assumption form $g$.
In other words, we must know that the proof is ``from $g$'s point of view,'' and thus it is appropriate to assume $\varphi$ from the fact that $g$ believes it.

The next two rules are the introduction and elimination rules for the $\believes{A}{}$ modality.
The first says that we can prove $\believes{A}{\varphi}$ if we can prove $\varphi$ with an $A$-labeled lock.
Since we can view locking a context as taking on $A$'s point of view, this says that in order to prove that $A$ believes $\varphi$, we can prove $\varphi$ from $A$'s point of view.
The elimination rule is more complicated.
Ignoring the role of~$g_1$ for the moment, \textsc{BelievesE} says that if you can prove $\believes{g_2}{\varphi}$, you may use the assumption $\varphi \at g_2$ to prove $\psi$.
Here, we're using a new piece of shorthand: $\believes{g}{\varphi}$ stands for a stack of modalities on top of $\varphi$, one for each part of $g$.
Thus, this says that if you can prove that $g_2$ believes $\varphi$, and you can prove $\psi$ assuming that $g_2$ believes $\varphi$, then you can prove $\psi$ immediately.
Adding in $g_1$ allows for more-flexible use of assumptions in the proof of $\believes{g_2}{\varphi}$, but necessitates adding that flexibility as requirement on the proof of $\psi$.

Finally, the last two rules say that $A$ has perfect knowledge about their own beliefs.
First, \textsc{SelfDown} says that if $A$ believes that $A$ itself believes something, it is true.
Second, \textsc{SelfUp} says that if $A$ believes something, then they believe that they believe it.
In the context of beliefs, these both make perfect sense; as we will see in Section~\ref{sec:lang-hier-chor}, we will need to generalize this in order to match many communication topologies which we care about in practice.

\paragraph{From Doxastic to Authorization Logic}
Authorization logics are doxastic logics with a notion of trust and communication among trusted lines.
For our purposes, authorization logic is parameterized by an additional preorder between agents, called \textsf{speaksfor}.
Then, we add one rule to our doxastic logic, called \textsc{SpeaksforE} in Figure~\ref{fig:dox-rules}.
This rule says that if $A$ speaks for $B$, and $A$ believes $\varphi$ (often written $A \mathrel{\textsf{says}} \varphi$ in authorization logics), then $B$ also believes $\varphi$.
Intuitively, if $A$ speaks for $B$, then $A$ is willing to send its evidence for $\varphi$ to $B$, and $B$ is willing to incorporate $A$'s evidence into its own beliefs.
Thus, we can think of \textsc{SpeaksforE} as representing a message send in a concurrent system, very similar to the choreographic send-and-receive operation.


%%% Local Variables:
%%% mode: LaTeX
%%% TeX-master: "conference"
%%% End:
