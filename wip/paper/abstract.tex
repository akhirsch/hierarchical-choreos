Functional choreographic programming suggests a new propositions-as-types paradigm might be possible.
In this new paradigm, communication is not modeled linearly; instead, ownership of a piece of data is modeled as a modality, and communication changes that modality.
However, we must find an appropriate modal logic for the other side of the propositions-as-types correspondence.
This paper argues for doxastic logics, or logics of belief.
In particular, authorization logics---doxastic logics with explicit communication---appear to represent \emph{hierarchical} choreographic programming.
This paper introduces hierarchical choreographic programming and presents Corps, a language for hierarchical choreographic programming with a propositions-as-types interpretation in authorization logic.
%%% Local Variables:
%%% mode: latex
%%% TeX-master: "conference"
%%% End:
