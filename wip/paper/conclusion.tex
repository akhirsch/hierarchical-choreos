\section{Conclusion}
\label{sec:conclusion}

In this work, we presented Corps, a language for \emph{hierarchical choreographic programming} with a propositions-as-types connection to authorization logic and doxastic logics more broadly.
While Corps is still very much an early work in progress, we believe that there are significant early results suggesting the fundamental nature of this work.
Moreover, should our conjectures hold, we can use them to translate between results about the type theory of functional choreographic programs and the proof theory of authorization logic.
Since the complexity of authorization-logic proof theory has prevented much development in the field, we are hopeful that this will spurn much development in both fields.

We hope that this work will inspire changes in how people design functional choreographic programming languages.
When restricted to only ``first-tier'' agents, Corps programs are close to  other functional choreographic programming languages, like Pirouette~\cite{HirschG22} and Chor$\lambda$~\cite{GraversenHM23,CruzFilipeGLMP21}.
However, subtle differences coming from the logical side have begun to appear; by exploiting those, we believe that we can design more grounded, foundational choreographic languages.

Finally, we note that this work suggests a deep connection between linear and doxastic logics: since deadlock-free communication seems to spring from both linear logic (via session types) and doxastic logics (via hierarchical choreographic programming).
The nature of this connection is unclear; each proof in doxastic logic seems to give rise to a series of linear-logic proofs, one for each projected program.
However, given that no-one has as of yet been able to show that projections of choreographic programs follow a linear-logic session type, this is unclear.
Nevertheless, tugging on this thread may lead to ideas that will change our understanding of both logic and concurrency.

%%% Local Variables:
%%% mode: latex
%%% TeX-master: "conference"
%%% End:
