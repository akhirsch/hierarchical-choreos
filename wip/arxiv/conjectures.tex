\section{Conjectures}
\label{sec:conjectures}

In current work we are exploring endpoint projection for Corps.
While in previous works processes were in one-to-one correspondence with agents, in Corps processes correspond to \emph{generalized} agents.
Thus, each node at every level of the tree is a process.
Endpoint projection is then a type-directed translation $\llbracket \Gamma \vdash e : \tau \rrbracket_{g}$, which is non-trivial only when $g$ is at or above $\locks{\Gamma}$.

We are also currently developing a normalizing semantics for Corps, in the usual sense of $\lambda$-calculus.
However, the connection to logic suggests that there are two notions of normalization that will be interesting to study: one which does not reduce any communications, and one which reduces communications of positive forms.
The first corresponds to cut elimination in sequent-calculus presentations of authorization logic, while the second corresponds to a stronger normal form of proofs where only variables may be communicated.
Both \citet{HirschACAT20} and \citet{GratzerNKB20} have found that this stronger normal form is useful when studying modal logics with communication; this suggests a deep reason why.
We plan to use a labeled-transition semantics of Corps which reduces communication to show that authorization and doxastic proofs enjoy this stronger normalization property.
The correctness of endpoint projection and normalization also suggest that Corps is deadlock free.

In authorization logics, an important property is \emph{noninterference}: if $A \mathrel{\textsf{speaksfor}} B$ is \emph{not} true, then $A$'s beliefs cannot affect $B$'s beliefs~\cite{HirschACAT20,HirschC13,GargP06,JiaVMZZSZ08}.
We believe a similar property is true for Corps, taking into account the \canupname{} and \candownname{}~relations in addition to \cansendname{}.
However, we believe that this will be significantly easier to prove for Corps, taking advantage of proof techniques for noninterference in information-flow control systems~\cite{MenzHLG23,SilverHCHZ23,HirschC21}.
We hope to leverage these proof techniques to allow for noninterference proofs for more complicated authorization logics in the future, as previous attempts have been stymied by the complex proof-theoretic constructions involved.



%%% Local Variables:
%%% mode: LaTeX
%%% TeX-master: "conference"
%%% End:
