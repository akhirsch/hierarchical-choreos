% Notes
\newcommand{\newcommenter}[3]{%
  \newcommand{#1}[1]{%
    \textcolor{#2}{\small\textsf{[#3: {##1}]}}%
  }%
}
\newcommenter{\akh}{purple}{AKH}
\newcommenter{\todo}{red}{TODO}

% AMSThm Setup
\theoremstyle{theorem}
\newtheorem{thm}{Theorem}
\newtheorem{lem}{Lemma}
\newtheorem{cor}{Corollary}
\theoremstyle{definition}
\newtheorem{defn}{Definition}
\newtheorem{ex}{Example}

\newcommand{\join}{\ensuremath{\mathbin{+\!\!+}}}

\newcommand{\believes}[2]{\ensuremath{\mathop{\square_{#1}} #2}}

\newcommand{\uu}{\ensuremath{(\,)}}

\newcommand{\letl}[5]{\textsf{let}_{#1}\;#2.#3 \mathrel{\coloneqq} #4 \mathrel{\textsf{in}} #5}

\newcommand{\fun}[3]{\textsf{fun}\;#1(#2) \mathrel{\coloneqq} #3}
\newcommand{\lam}[2]{\lambda#1.\;#2}
\newcommand{\appfun}[2]{#1\;#2}

\newcommand{\projname}{\textsf{proj}}
\newcommand{\projl}[1]{\appfun{\projname_1}{#1}}
\newcommand{\projr}[1]{\appfun{\projname_2}{#1}}

\newcommand{\inl}[1]{\appfun{\textsf{inl}}{#1}}
\newcommand{\inr}[1]{\appfun{\textsf{inr}}{#1}}
\newcommand{\case}[5]{\ensuremath{\textsf{case}\;#1 \mathrel{\textsf{of}} \inl{#2} \Rightarrow #3;\; \inr{#4} \Rightarrow #5}}

\newcommand{\casev}[1]{\ensuremath{\textsf{case}\;#1 \mathrel{\textsf{of}} \_ \Rightarrow \bot}}

\newcommand{\send}[2]{#1 \mathbin{\leadsto} #2}

\newcommand{\upname}{\textsf{up}}
\newcommand{\up}[2]{\appfun{\upname_{#1}}{#2}}
\newcommand{\downname}{\textsf{down}}
\newcommand{\down}[2]{\appfun{\downname_{#1}}{#2}}

\newcommand{\unit}{\textsf{unit}\xspace}
\newcommand{\void}{\textsf{void}\xspace}

\newcommand{\lock}{\text{\faLock}}
\newcommand{\at}{\ensuremath{\mathrel{@}}}
\newcommand{\locks}[1]{\text{locks}(#1)}

\newcommand{\reps}[2]{#1 \mathrel{\vartriangleright} #2}
\newcommand{\cansendname}{\text{CanSend}}
\newcommand{\cansend}[2]{\cansendname(#1, #2)}
\newcommand{\candownname}{\text{CanDown}}
\newcommand{\candown}[2]{\candownname(#1, #2)}
\newcommand{\canupname}{\text{CanUp}}
\newcommand{\canup}[2]{\canupname(#1, #2)}



%%% Local Variables:
%%% mode: latex
%%% TeX-master: "conference"
%%% End:
