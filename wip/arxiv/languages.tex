\section{A Language for Hierarchical Choreographic Programming}
\label{sec:lang-hier-chor}

\begin{figure}
  \centering
  \begin{syntax}
    \category[Types]{\tau}
    \alternative{\unit}
    \alternative{\void}
    \alternative{\believes{A}{\tau}}
    \alternative{\tau_1 \times \tau_2}
    \alternative{\tau_1 + \tau_2}
    \alternative{\tau_1 \to \tau_2}
    \category[Expressions]{e}
    \alternative{x}
    \alternative{A.e}
    \alternative{\letl{g_1}{g_2}{x}{e_1}{e_2}}\\
    \alternative{\send{e}{A}}
    \alternative{\up{g}{e}}
    \alternative{\down{g}{e}}
  \end{syntax}
  \begin{mathpar}
    \infer*[left=Axiom]{
      \locks{\Delta} = g
    } {
      \Gamma, x : \tau @ g, \Delta \vdash x : \tau
    }\\
    \infer*[left=BelievesI]{
      \Gamma, \lock_A \vdash e : \tau
    }{
      \Gamma \vdash A.e : \believes{A}{\tau}
    }\and
    \infer*[right=BelievesE]{
      \Gamma, \lock_{g_1} \vdash e_1 : \believes{g_2}{\tau}\\
      \Gamma, x : \tau \at g_1 \join g_2 \vdash e_2 : \sigma
    }{
      \Gamma \vdash \letl{g_1}{g_2}{x}{e_1}{e_2} : \sigma
    }\\
    \infer*[right=Down]{
      \Gamma \vdash e : \believes{g}{\tau}\\
      \candown{\locks{\Gamma}}{g}
    }{
      \Gamma \vdash \down{g}{e} : \tau
    }\and
    \infer*[left=Up]{
      \Gamma \vdash e : \tau\\
      \canup{\locks{\Gamma}}{g}
    }{
      \Gamma \vdash \up{g}{e} : \believes{g}{\tau}
    }\\
    \infer*[left=Send]{
      \Gamma \vdash e : \believes{g_1}{\tau}\\
      \cansend{g_1}{g_2}
    }{
      \Gamma \vdash \send{e}{g_2} : \believes{g_2}{\tau}
    }\and
  \end{mathpar}
  \caption{Corp Selected Syntax and Typing Rules}
  \label{fig:choreo-types}
\end{figure}

So far, we have focused on the purely logical presentation of doxastic and authorization logics.
We now present Corps (short for ``Corps de Ballet,'' a rank in ballet troupes), a language for \emph{hierarchical choreographic programming}.
The type syntax, along with selected pieces of the program syntax and the corresponding typing rules, can be found in Figure~\ref{fig:choreo-types}.
Every type connective has its standard interpretation in logic.
For instance, the type connective~$\times$ corresponds to the logical connective~$\land$.
The only exception is $\believes{A}{\tau}$, which represents a $\tau$ stored on the agent~$A$ located one level up in the hierarchy.

We can view a typing judgment $\Gamma \vdash e : \tau$ as saying that $e$ computes a $\tau$ on the process $\locks{\Gamma}$.
From this perspective, the \textsc{Axiom} rule says that we can use $x$ as a $\tau$ if the current process has access to a $\tau$ named $x$ in the context.
Furthermore, the \textsc{BelievesI}~rule says that to compute a $\believes{A}{\tau}$ on the process $\locks{\Gamma}$, we need to compute a $\tau$ on $\locks{\Gamma} \cdot A$.
The \textsc{BelievesE}~rule says that if we can compute a $\believes{g_2}{\tau}$ on the process $g_1$, then we can use the result of that computation as a $\tau$ on the process $g_1 \join g_2$.

The remaining rules correspond not only to similar rules in logic, but to explicit communication.
The rule~\textsc{Down} corresponds to the rule~\textsc{SelfDown} in doxastic logic.
However, as we can see here, it is more general than the original \textsc{SelfDown}~rule.
Just as authorization logic is parameterized by the \textsf{speaksfor} relation between agents to determine when they can communicate, Corps is parameterized by three similar relations between generalized agents.
Here, the relation~$\candownname$ determines when communication is allowed \emph{down the tree}.
To recover the \textsc{SelfDown}~rule, we can set $\candown{g_1}{g_2} \overset{\Delta}{\iff} g_1 = g \cdot A \land g_2 = g \cdot A \cdot A$; that is, $g_2$ is allowed to communicate down to $g_1$ only if $g_1$ is some agent~$A$ (possibly up the tree by some path), and $g_2$ is the agent named~$A$ directly above $g_1$.
This allows communication down the tree only from $A$'s beliefs about itself.
Similarly, the rule~\textsc{Up} corresponds to the original \textsc{SelfUp}~rule, but generalized with the $\canupname$~relation controlling when communication is allowed up the tree.
Finally, the rule~\textsc{Send} corresponds directly to the original \textsc{SpeaksforE}~rule, with \cansendname{} corresponding to \textsf{speaksfor}.

By setting different meanings for \candownname{}, \canupname{}, and \cansendname{}, we can get very different topologies all within the hierarchical style.
For instance, to get the traditional fully-connected topology between sibling nodes for choreographies, we need only set $\cansend{g_1}{g_2} \overset{\Delta}{\iff} \textsf{True}$.
Many of the traditional settings for these relations only involve a postfix of the generalized agent, similar to the traditional setting for \candownname{} discussed earlier.
This creates a uniform set of policies: every agent allows its children of the same name to communicate both with itself and with each other.
However, the generality of these relations allows for different (generalized) agents to have different policies.
We expect that many topologies that arise in practice will not match the requirement that a node named~$A$ only communicate with a thread also named~$A$, leading to this generalization.


%%% Local Variables:
%%% mode: LaTeX
%%% TeX-master: "conference"
%%% End:
